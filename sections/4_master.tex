\newpage
\section{Zap Master - {\ttfamily "master"}}
\label{sec:zapmaster_predicate}


\subsection{Overview}
The ZapWallet Master is a predicate that provides secure asset custody through modular validation. The master serves as the authority to validate
ZapWallet transactions, initialization (if paying own gas) and wallet upgrades.

\subsection{Definitions}

\textbf{Basic Types}\\
Let:
\begin{itemize}
    \item $\mathbb{B}_{256}$ be the set of 256-bit values
    \item $\mathbb{A}$ be the set of valid Fuel addresses
\end{itemize}

\textbf{Module}\\
A module $M$ is defined as a tuple:
\[ M = (a, p) \text{ where } a \in \mathbb{B}_{256}, p \in \mathbb{A} \]
where:
\begin{itemize}
    \item $a$ is the unique AssetId of the module
    \item $p$ is the predicate address of the module
\end{itemize}

\textbf{Wallet Configuration}\\
A wallet configuration $W$ consists of:
\[ W = (M_0, M_1, ..., M_8, o) \]
where:
\begin{itemize}
    \item $M_0$ is the upgrade module
    \item $M_1$ through $M_8$ are the operational modules
    \item $o \in \mathbb{B}_{256}$ is the owner's address
\end{itemize}




% ----------------------------------------------------------------------------------------------------
\subsection{Transaction Types and Validation}

The ZapWallet master predicate's \texttt{main()} function accepts an optional \texttt{WalletOp} parameter that serves two distinct purposes depending on the transaction type:

\begin{enumerate}
    \item \textbf{Initialization:} When present (\texttt{Some(WalletOp)}), validates wallet initialization
    \item \textbf{Module Operations:} When absent (\texttt{None}), validates module-based transactions
\end{enumerate}




\subsubsection{Initialization}
During initialization, \texttt{WalletOp} contains:

\begin{lstlisting}
pub struct WalletOp {
    pub evm_addr: b256,      // padded ETH address of wallet owner
    pub compsig: Bytes,      // Compact signature
    pub command: String,     // Initialization command
}
\end{lstlisting}



An initialization transaction $T_i$ must satisfy:
\[ V_i(T_i, s, o) = true \]
where:
\begin{itemize}
    \item $s$ is a valid EIP712 signature
    \item $o$ is the owner's address
    \item $V_i$ is the initialization verification function
\end{itemize}

$V_i$ verifies:
\begin{enumerate}
    \item Exactly two inputs: one coin and one contract
    \item Valid change output
    \item Valid signature recovering to owner
    \item No module assets present
\end{enumerate}





\subsubsection{Module Operations}
For all other operations, transaction validation relies on:
\begin{itemize}
    \item Presence of exactly one module asset in inputs
    \item Proper return of module asset in outputs
    \item Module-specific validation logic
    \item Nonce token accounting (when required)
\end{itemize}

A module operation transaction $T_m$ must satisfy:
\[ V_m(T_m, M_i) = true \]
for exactly one module $M_i$ where $i \in \{1,...,8\}$

$V_m$ verifies:
\begin{enumerate}
    \item Exactly one module asset present in inputs
    \item Module asset returned correctly in outputs
    \item Module-specific validation passes
\end{enumerate}



\textbf{Upgrade Operation}\\
An upgrade transaction $T_u$ must satisfy:
\[ V_u(T_u, M_0) = true \]

$V_u$ verifies:
\begin{enumerate}
    \item Only module $M_0$ present
    \item Upgrade module asset handled correctly (sent to ZapManager)
    \item No other module assets present
\end{enumerate}



This dual-purpose design allows the \texttt{WalletOp} to handle initialization while remaining unintrusive for regular module operations.\\





\subsection{State Transitions}

\textbf{Module State Vector}\\
For any transaction T, let $\vec{m}$ be a boolean vector where:
\[ \vec{m} = [m_0, m_1, ..., m_8] \text{ where } m_i \in \{0,1\} \]
indicating the presence (1) or absence (0) of each module in the transaction inputs.\\

\textbf{Valid States}\\
A transaction is valid if and only if $\vec{m}$ satisfies exactly one of:
\begin{enumerate}
    \item Initialization: $\vec{m} = [0,0,...,0]$
    \item Single Module: $\vec{m}$ contains exactly one 1
    \item Upgrade: $\vec{m} = [1,0,...,0]$
\end{enumerate}



\subsection{Security Properties}

\textbf{Module Isolation}\\
For any valid transaction T:
\[ |\{i : m_i = 1\}| \leq 1 \]
Meaning no more than one module can be active in a single transaction.\\

\textbf{Asset Conservation}\\
For any module $M_i$ present in transaction inputs:
\[ \exists \text{ output } o : o.asset = M_i.a \land o.amount = 1 \land o.to = M_i.p \]
Meaning any module asset must be properly returned to its predicate.\\

\textbf{State Consistency}\\
The master predicate enforces:
\begin{itemize}
    \item No double-module usage
    \item Proper initialization sequence
    \item Upgrade isolation
    \item Module asset conservation
\end{itemize}


\subsubsection{Implementation Notes}

\textbf{Module Detection}\\
The system identifies modules by matching both:
\begin{itemize}
    \item AssetId ($\mathbb{B}_{256}$)
    \item Predicate Address ($\mathbb{A}$)
\end{itemize}

\textbf{Validation Flow}\\
1. Scan inputs for module assets
2. Build module state vector
3. Determine transaction type
4. Apply appropriate validation rules
5. Verify output conditions



% ---------------------------------------------------------------------------------------------------------------

\subsection{Initialization Flow}
\label{subsec:initialization:sec:zapmaster_predicate}
The initialization of a ZapWallet can be done in one of two ways; using the owners Fuel \text{BASE\_ASSET} to pay for gas OR from a third party that spends their own gas. If the owner is
initializing their own ZapWallet (self-initialization) this involves spending a \text{BASE\_ASSET} UTXO from the master, which requires a signature from the owner within
the \texttt{WalletOp} parameter of the master \text{main()} function call parameters. An initialization by way of a third party does not spend gas (or any other asset) from the
owners master.

\subsubsection{Self-Initialization}
A valid self-initialization initialization transaction requires:
\begin{itemize}
    \item Exactly two inputs:
        \begin{itemize}
            \item One coin input containing FUEL \text{BASE\_ASSET} for gas from the owners master address
            \item One contract input referencing the ZapManager contract
        \end{itemize}
    \item A single change output returning unused FUEL \text{BASE\_ASSET} to owners master address
\end{itemize}


The self-initialization route requires a single distinct signature from the owner. A standard EIP-712 compliant signature over the initialization data structure:
% \begin{small}
% \begin{verbatim}
% Initialization(
%     string command,    // "ZapWalletInitialize"
%     bytes32 evmaddr,   // Owner's padded EVM address
%     bytes32 utxoid     // UTXO ID of the gas coin input
% )
% \end{verbatim}
% \end{small}

\begin{lstlisting}
    Initialization(
        string command,    // "ZapWalletInitialize"
        bytes32 evmaddr,   // Owner's padded EVM address
        bytes32 utxoid     // UTXO ID of the gas coin input
    )
\end{lstlisting}


\subsubsection{Third Part Initialization}
A valid initialization transaction made my a third party requires:
\begin{itemize}
    \item Exactly two inputs:
        \begin{itemize}
            \item One coin input containing FUEL \text{BASE\_ASSET} for gas from whomever is sponsoring the transaction.
            \item One contract input referencing the ZapManager contract
        \end{itemize}
    \item A single change output returning unused FUEL \text{BASE\_ASSET} to the transaction sender
\end{itemize}







% --- calling the initialize_wallet() function at the ZapManager
\subsubsection{Contract Interaction}
The initialization flow proceeds as follows:
\begin{enumerate}
   \item The transaction calls the ZapManager contract function
    \begin{lstlisting}
    initialize_wallet()
    \end{lstlisting}

   \item The contract verifies no existing nonce token exists for the tuple $(a, m)$ where:
       \begin{itemize}
           \item $a \in \mathbb{B}_{256}$: The padded EVM address
           \item $m \in Address$: The master predicate address
       \end{itemize}
       See Section~\ref{subsec:overview-sec:zapmanager_contract} for details regarding the mapping.
   \item Upon verification, the contract:
       \begin{itemize}
           \item Mints the nonce tokens
           \item Sends (\text{NONCE\_MAX} - 1) to the ZapWallet master predicate
           \item Retains one token for transaction validation
       \end{itemize}
   \item The master predicate verifies:
       \begin{itemize}
           \item The initialization signature
           \item Transaction structure
           \item Change output validity
       \end{itemize}
\end{enumerate}




Upon successful completion of the \text{initialize\_wallet} function call, the ZapWallet is initialized and ready for module operations.