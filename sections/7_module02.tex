\newpage
\section{Module 02 - {\ttfamily ""module02\_evm\_txtype2"}}
\label{sec:module02_predicate}


\subsection{Overview}
Module 02 handles EIP-1559 EVM transaction validation within the ZapWallet system. Like Module 01, it processes \text{BASE\_ASSET}
transfers but implements the EIP-1559 fee market mechanism instead of Legacy EVM's first-price gas auction. For transaction processing
details including RLP decoding and address mapping, see sections \ref{subsec:rlp_decoding} and \ref{sec:module01_predicate}.\\

\subsection{Key Differences from Module 01}
The primary distinction lies in the fee structure and account of the \text{max\_tx\_cost} signed by the owner:\\

\textbf{Legacy (Module 01):}
\[ gas\_cost = gasPrice \times gasLimit \]

\textbf{EIP-1559 (Module 02):}
\[ gas\_cost = (baseFee + maxPriorityFeePerGas) \times gasLimit \]
where:
\[ maxFeePerGas \geq baseFee + maxPriorityFeePerGas \]


% % ----------------------------------------------------------------------------------------------
% %


\subsection{Gas Calculation and Cost Flow}
\label{subsec:gas_costing:sec:module02_predicate}
Module 02 implements a flexible gas fee structure that handles network fees and optional builder tips. The cost structure is derived from the
gas parameters in the EIP-1559 transaction:\\

\subsubsection{Gas Parameters}
For a transaction with gas parameters $(g_{limit}, g_{price})$:
\[ max\_tx\_cost = g_{limit} \times g_{price} \]

\subsubsection{Cost Components}
The total cost budget consists of:
\begin{itemize}
    \item \textbf{Network fee:} Required cost for transaction execution
    \item \textbf{Builder tip:} Optional incentive for transaction inclusion
    \item \textbf{Change:} Excess funds returned to sender
\end{itemize}

For input amount $i$, receiver value $v$, network fee $f$, and builder tip $b$:
\begin{align*}
i &\geq v + max\_tx\_cost \\
0 &\leq b \leq max\_tx\_cost - f \\
change &= i - (v + f + b)
\end{align*}

\begin{figure}[H]
    \centering
    \begin{tikzpicture}[
        node distance=1.5cm,
        box/.style={rectangle, draw=black, thick, minimum width=3cm, minimum height=1cm, align=center},
        group/.style={rectangle, draw=black, dashed, inner sep=0.5cm}
    ]
        % RLP Transaction group
        \begin{scope}[local bounding box=rlp]
            \node[box] (G) {Gas Limit};
            \node[box] (P) [right=1cm of G] {Gas Price};
        \end{scope}
        \node[above=0.3cm of rlp] {RLP Transaction};
        \draw [dashed] ($(rlp.north west)+(-0.2,0.2)$) rectangle ($(rlp.south east)+(0.2,-0.2)$);

        % Calculation group
        \begin{scope}[local bounding box=calc]
            \node[box] (M) [below=2cm of rlp] {max\_tx\_cost = \\ limit $\times$ price};
        \end{scope}
        \node[above=0.3cm of calc] {Calculations};
        \draw [dashed] ($(calc.north west)+(-0.2,0.2)$) rectangle ($(calc.south east)+(0.2,-0.2)$);

        % Constraints group
        \begin{scope}[local bounding box=const]
            \node[box] (C1) [below=2cm of M] {network\_fee $\leq$ max\_tx\_cost};
            \node[box] (C2) [right=1cm of C1] {builder\_tip $\leq$ \\ max\_tx\_cost - network\_fee};
            \node[box] (C3) [below=1cm of C1] {change = input - value - fee - tip};
        \end{scope}
        \node[above=0.3cm of const] {Constraints};
        \draw [dashed] ($(const.north west)+(-0.2,0.2)$) rectangle ($(const.south east)+(0.2,-0.2)$);

        % Arrows
        \draw[-stealth] (G) -- (M);
        \draw[-stealth] (P) -- (M);
        \draw[-stealth] (M) -- (C1);
        \draw[-stealth] (M) -- (C2);
        \draw[-stealth] (C1) -- (C3);
        \draw[-stealth] (C2) -- (C3);
    \end{tikzpicture}
    \caption{Gas Parameters and Cost Flow}
    \label{fig:gas-params-tikz}
\end{figure}

This structure ensures:
\begin{itemize}
    \item Sufficient input covers value transfer and maximum costs
    \item Network fee payment is guaranteed
    \item Builder tips are optional but bounded
    \item Excess funds automatically return to sender
\end{itemize}




% % ----------------------------------------------------------------------------------------------
% %

\subsection{Transaction Structure}
\begin{figure}[H]
    \scalebox{0.85}{
    \centerline{
    \xymatrix@R=2.5pc@C=1.5pc{
        & \txt{Input type\\ Asset type\\ Amount}
        & \txt{Verification\\ Signature \\ type}
        & \txt{Output type\\ Receiver\\ Asset type\\ Amount} \\
        & *+[F]\txt{Coin predicate:\\ Module 02\\ asset: Module 02\\ AssetId\\ amount: 1}
            \ar[r] & *+[H]\txt{EIP-1559\\ Transaction}
            \ar[r] & *+[F]\txt{type: Coin\\ to: Module 02\\ asset: Module 02\\ AssetId\\ amount: 1} \\
        & *+[F]\txt{Coin predicate:\\ Master\\ asset: Nonce AssetId\\ amount: $n$,\\ where $n \geq 1$}
            \ar[r] & *+[H]\txt{Nonce value \\ \& ownership\\ validation}
            \ar[r] & *+[F]\txt{type: Coin\\ to: Master\\ asset: Nonce AssetId\\ amount: $n - 1$} \\
        & *+[F]\txt{Coin predicate:\\ Master\\ asset: BASE\_ASSET\\ amount:\\ value + max\_tx\_cost}
            \ar[r] & *+[H]\txt{Value \&\\ max\_tx\_cost \\ Validation}
            \ar[r] & *+[F]\txt{type: Coin\\ to: Receiver\\ asset: BASE\_ASSET\\ amount: value} \\
        & \phantom{*+[F]\txt{Placeholder}}
            & \phantom{*+[H]\txt{Placeholder}}
            & *+[F.:purple:dashed]\txt{type: Coin\\ to: Transaction\\ Builder\\ asset: BASE\_ASSET\\ amount:\\ priority fee} \\
        & \phantom{*+[F]\txt{Placeholder}}
            & \phantom{*+[H]\txt{Placeholder}}
            & *+[F.:grey:dashed]\txt{type: Change\\ to: Master\\ asset: BASE\_ASSET\\ amount: remaining}
    }}
    }
    \caption{Module 02 EIP-1559 Transaction Flow and Verification}
    \label{fig:eip1559-flow-xy}
\end{figure}





\subsection{Security Properties}
Module 02 maintains the same security properties as Module 01:

\subsubsection{Asset Conservation}
For module02 asset $M_2$ and predicate address $P$:
\[ \exists \text{ output } o : o.asset = M_2 \land o.amount = 1 \land o.to = P \]

\subsubsection{Signature Replay Prevention}
As in other modules, the nonce mechanism protects against signature replay attacks:
\[ n = NONCE\_MAX - t \]
\[ \exists \text{ output } o : o.asset = N \land o.amount = n - 1 \]

\subsubsection{Value Transfer}
For transaction value $v$ and max cost $c$:
\[ \sum \text{inputs} \geq v + c \]
\[ \exists \text{ output } o : o.asset = BASE\_ASSET \land o.amount = v \land o.to = receiver \]

All transaction validations and asset flow mechanisms remain identical to Module 01, with the only difference being the
fee calculation and structure in the RLP transaction data.


\subsection{Error Handling}
The module includes error handling that propagates errors to at maximum the \text{main()} scope. However as predicates
must return boolean results, all errors result in a \text{false} returned from the main execution.




