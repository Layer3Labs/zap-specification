\newpage
\section{Module 01 - {\ttfamily "module01\_evm\_txtype1"}}
\label{sec:module01_predicate}

% \subsection{Overview}
% Module 01 handles Legacy EVM transaction validation within the ZapWallet system. It validates and processes EVM transactions
% by decoding RLP-encoded transaction data, verifying signatures, and ensuring proper asset flow through the Fuel network. Module0 01
% only works for Fuel \text{BASE\_ASSET} transfers and the receiver should be the EVM address owner of a ZapWallet mapping.


\subsection{Overview}
Module 01 handles Legacy EVM transaction validation within the ZapWallet system. It validates and processes EVM transactions
by decoding RLP-encoded transaction data, verifying signatures, and ensuring proper asset flow through the Fuel network. Module 01
only works for Fuel \text{BASE\_ASSET} transfers. The receiver address in the RLP transaction data is an EVM address which maps
to a corresponding ZapWallet master predicate address. \\

Formally, for a transaction $T$ with recipient EVM address $r_{evm}$, the receiving ZapWallet address $w$ is derived:

\[ f: r_{evm} \rightarrow w \text{ where } w = P(r_{evm}, c) \]

where:
\begin{itemize}
   \item $P$ is the predicate address calculation function
   \item $c$ is the receiver's wallet bytecode
   \item $w$ is the resulting ZapWallet master predicate address
\end{itemize}

This mapping and calculation of the receivers address insures that EVM-style transactions are compatible with with ZapWallet's and
by extension the Fuel network, while maintaining proper asset custody.



% % ----------------------------------------------------------------------------------------------
% Merkle and address mapping

\subsection{Receiver Address Mapping}
For each transaction, Module 01 must verify that the Fuel \text{BASE\_ASSET} is sent to the correct ZapWallet address corresponding
to the EVM address specified in the RLP transaction data. This mapping is accomplished through a merkle tree-based predicate address calculation.

\subsubsection{Merkle Tree Structure}
The ZapWallet master predicate address is derived from a two-leaf merkle tree where:
\begin{itemize}
    \item Left Leaf ($L_1$): Fixed blueprint hash known at compilation
    \item Right Leaf ($L_2$): Variable leaf containing receiver's EVM address and configurables for the receiver's master predicate.
\end{itemize}

% The derivation process follows:
% \[ H(L_2) = \text{SHA256}(0 \parallel bytecode[EVM_{addr}]) \]
% \[ N = \text{SHA256}(1 \parallel L_1 \parallel H(L_2)) \]
% \[ root = N \]
% \[ address = \text{SHA256}(\texttt{FUEL} \parallel root) \]

% where:
% \begin{itemize}
%     \item $0$ is the leaf prefix byte
%     \item $1$ is the node prefix byte
%     \item $\texttt{FUEL}$ is the contract ID seed (\texttt{0x4655454C})
%     \item $EVM_{addr}$ is placed at position 6760 in the right leaf
% \end{itemize}

The derivation process follows these precise steps:

\textbf{1. Leaf Hashing:}
\[ H(L_2) = \text{SHA256}(\texttt{0x00} \parallel bytecode[EVM_{addr}]) \]
where \texttt{0x00} is the leaf prefix identifying this as a leaf node

\textbf{2. Node Computation:}
\[ N = \text{SHA256}(\texttt{0x01} \parallel L_1 \parallel H(L_2)) \]
where \texttt{0x01} is the node prefix identifying this as an internal node

\textbf{3. Root:}
\[ root = N \]

\textbf{4. Predicate Address:}
\[ address = \text{SHA256}(\texttt{0x4655454C} \parallel root) \]
where \texttt{0x4655454C} represents "FUEL" in ASCII as the contract ID seed. For a complete
implementation of Fuel's merkle tree specification, see the official implementation
at \url{https://github.com/FuelLabs/fuel-vm/tree/master/fuel-merkle}.


\subsubsection{Verification Process}
For a transaction with recipient EVM address $r_{evm}$, Module 01:
1. Takes the provided right leaf bytecode
2. Inserts $r_{evm}$ at the predetermined position (6760)
3. Calculates the merkle tree and resulting predicate address
4. Verifies the \text{BASE\_ASSET} output is sent to this address

This ensures that funds from EVM-style transactions are always directed to the correct ZapWallet predicate address,
maintaining the mapping between EVM addresses and their corresponding ZapWallets.






% % ----------------------------------------------------------------------------------------------
% RLP

\subsection{RLP Transaction Decoding}
\label{subsec:rlp_decoding}

EVM transactions are encoded using Recursive Length Prefix (RLP) encoding (previous to simple serialize (SSZ)). Modules 01-03 implement
RLP decoding systems for Legacy and EIP-1559 transactions (which can include ERC20 transfers). This RLP decoding scheme
is similar across Modules 01, 02 and 03 in the ZapWallet and can be broken down into several layers:

\subsubsection{RLP Encoding Rules}
The decoder follows standard Ethereum RLP encoding rules where bytes are categorized by their first byte (prefix):
\begin{itemize}
    \item \texttt{[0x00-0x7f]}: Single byte (transaction type identifier)
    \item \texttt{[0x80-0xb7]}: String with length 0-55 bytes
    \item \texttt{[0xb7-0xbf]}: String with length $>$ 55 bytes
    \item \texttt{[0xc0-0xf7]}: List with total payload 0-55 bytes
    \item \texttt{[0xf7-0xff]}: List with total payload $>$ 55 bytes
\end{itemize}


% -------------------------------------------------------------------------
% Legacy

\subsubsection{Legacy EVM Transaction Structure}
A Legacy EVM transaction's RLP encoding contains the following fields in order:
\begin{lstlisting}
[
    nonce,           // u64: Transaction sequence number
    gasPrice,        // u64: Price per unit of gas
    gasLimit,        // u64: Maximum gas allowed
    to,              // b256: Recipient address
    value,           // u256: Amount of ETH to transfer
    chainId,         // u64: Network identifier (EIP-155)
    r,               // b256: Signature component
    s                // b256: Signature component
]
\end{lstlisting}

\subsubsection{Decoding Process}
The decoding process follows these steps:

1. \textbf{Payload Identification}:
\begin{itemize}
    \item Reads the first byte to determine the encoding type
    \item Calculates total payload length
    \item Validates encoding structure
\end{itemize}

2. \textbf{Field Extraction}:
\[ decode: \text{RLP} \rightarrow (field, ptr, len) \]
where:
\begin{itemize}
    \item $field$ is the decoded value
    \item $ptr$ is the new buffer position
    \item $len$ is the field length
\end{itemize}

3. \textbf{Type Conversion}:
Different utilities handle specific data types:
\begin{itemize}
    \item \texttt{rlp\_read\_u64}: Converts bytes to unsigned 64-bit integer
    \item \texttt{rlp\_read\_b256}: Converts bytes to 256-bit value
    \item \texttt{rlp\_read\_bytes\_to\_u256}: Converts variable-length bytes to u256
\end{itemize}

\subsubsection{EIP-155 Processing}
The decoder implements EIP-155 chain ID recovery:
\[ chain\_id = \begin{cases}
    \frac{v - 35}{2} & \text{if } v \geq 35 \\
    0 & \text{otherwise}
\end{cases} \]

where $v$ is the recovery identifier from the signature.

\subsubsection{Signature Recovery}
The final step reconstructs the compact signature:
\begin{itemize}
    \item Normalizes the recovery ID
    \item Combines $r$ and $s$ components
    \item Sets the recovery bit in the high bit of $s$
\end{itemize}


% -------------------------------------------------------------------------
% EIP-1559

\subsubsection{EVM EIP-1559 Transaction Structure}
An EVM EIP-1559 (Type 2) transaction's RLP encoding contains the following fields in order:
\begin{lstlisting}
[
    chainId,              // u64: Network identifier
    nonce,               // u64: Transaction sequence number
    maxPriorityFeePerGas,// u64: Max priority fee (tip)
    maxFeePerGas,        // u64: Maximum total fee per gas
    gasLimit,            // u64: Maximum gas allowed
    to,                  // b256: Recipient address
    value,              // u256: Amount of ETH to transfer
    data,               // bytes: Transaction data (empty for simple transfers)
    accessList,         // []: List of addresses and storage keys
    v,                  // u64: Signature recovery identifier
    r,                  // b256: Signature component
    s                   // b256: Signature component
]
\end{lstlisting}


% -------------------------------------------------------------------------
% EIP-1559 ERC20

\subsubsection{EVM ERC20 Transaction Structure}
An EVM ERC20 transaction (Type 2) extends the EIP-1559 structure with specific data field encoding:
\begin{lstlisting}
[
    chainId,              // u64: Network identifier
    nonce,               // u64: Transaction sequence number
    maxPriorityFeePerGas,// u64: Max priority fee (tip)
    maxFeePerGas,        // u64: Maximum total fee per gas
    gasLimit,            // u64: Maximum gas allowed
    to,                  // b256: ERC20 contract address
    value,              // u256: Must be 0 for ERC20 transfers
    data: [             // bytes: Encoded transfer function call
        methodId,        // bytes4: transfer(address,uint256) selector
        recipient,       // b256: Token EVM recipient address
        amount          // b256: Amount of tokens to transfer
    ],
    accessList,         // []: List of addresses and storage keys
    v,                  // u64: Signature recovery identifier
    r,                  // b256: Signature component
    s                   // b256: Signature component
]
\end{lstlisting}

\textbf{ERC20 Data Field Details:}
The data field for ERC20 transfers consists of:
\begin{itemize}
    \item \texttt{methodId}: \texttt{0xa9059cbb} (transfer function selector)
    \item \texttt{recipient}: 32-byte padded address
    \item \texttt{amount}: 32-byte token amount
\end{itemize}

The field decoded to the ERC20 contract address is translated to the corresponding Fuel Native Asset ID
using the \texttt{compare\_asset\_ids} function. This allows Fuel native assets to be transferred as EVM ERC20 transfers
on the Fuel network. See section \ref{subsec:asset_maapping:sec:module03_predicate} for more details.



% % ----------------------------------------------------------------------------------------------
% % ----------------------------------------------------------------------------------------------
% Module 01 functionality


\subsection{Module 01 Core Functionality}
The module validates transactions through several key steps:
\begin{enumerate}
    \item RLP decoding of signed EVM transactions
    \item Signature and chain ID verification
    \item Asset input and output validation
    \item Nonce management
    \item Gas price and limit verification
\end{enumerate}

\subsection{Transaction Processing}
\subsubsection{EVM Transaction Decoding}
The module decodes Legacy EVM transactions into their constituent fields:
\begin{itemize}
    \item Chain ID (per EIP-155)
    \item Nonce
    \item Gas Price
    \item Gas Limit
    \item Recipient EVM Address
    \item Value (Base Asset amount in Wei)
    \item Signature Components (v, r, s)
\end{itemize}

\subsubsection{Validation Requirements}
For a transaction to be valid, it must satisfy:
\begin{itemize}
    \item Correct chain ID matching Fuel network
    \item Valid signature from wallet owner
    \item Sufficient input amounts covering value, gas and builder\_tip
    \item Proper nonce accounting matching the RLP data
    \item Valid input and output coins with correct owners
\end{itemize}




\subsection{Asset Flow}
A transaction containing Module 01 processes three types of assets:
\begin{itemize}
    \item Module 01 asset (for validation authorization of the signed RLP data)
    \item Nonce asset (to prevent signature replay)
    \item \text{BASE\_ASSET} (for value transfer, gas and builder tip) from the master
\end{itemize}


The builder\_tip needs be a Coin Output with some value (can be zero). It is not enforced that the transaction builder
is required to "tip" themselves. Therefore, the value of the \text{BASE\_ASSET} output to the receiver is always the amount specified in the RLP
data. The gas consumed by the transaction comes out of the \text{max\_tx\_cost} budget, which is made up of:

\[
\text{max\_tx\_cost} = \text{network fee} + \text{builder\_tip}
\]

Any difference in \text{BASE\_ASSET} that is not consumed by the sum of the transaction outputs (for \text{BASE\_ASSET}) is sent back to the
owners ZapWallet master Address as a Change output. See section \ref{subsec:gas_costing:sec:module02_predicate} for further details.


\subsection{Transaction Structure}
\begin{figure}[H]
    \scalebox{0.85}{
    \centerline{
    \xymatrix@R=2.5pc@C=1.5pc{
        & \txt{Input type\\ Asset type\\ Amount}
        & \txt{Verification\\ Signature \\ type}
        & \txt{Output type\\ Receiver\\ Asset type\\ Amount} \\
        & *+[F]\txt{Coin predicate:\\ Module 01\\ asset: Module 01\\ AssetId\\ amount: 1}
            \ar[r] & *+[H]\txt{Legacy EVM\\ Transaction}
            \ar[r] & *+[F]\txt{type: Coin\\ to: Module 01\\ asset: Module 01\\ AssetId\\ amount: 1} \\
        & *+[F=]\txt{Coin predicate:\\ Master\\ asset: Nonce AssetId\\ amount: $n$,\\ where $n \geq 1$}
            \ar[r] & *+[H]\txt{Nonce value \\ \& ownership\\ validation}
            \ar[r] & *+[F=]\txt{type: Coin\\ to: Master\\ asset: Nonce AssetId\\ amount: $n - 1$} \\
        & *+[F=]\txt{Coin predicate:\\ Master\\ asset: BASE\_ASSET\\ amount:\\ value + max\_tx\_cost}
            \ar[r] & *+[H]\txt{Value \&\\ max\_tx\_cost \\ Validation}
            \ar[r] & *+[F=]\txt{type: Coin\\ to: Receiver\\ asset: BASE\_ASSET\\ amount: value} \\
        & \phantom{*+[F=]\txt{Placeholder}}
            \ar[r] & *+[H]\txt{max\_tx\_cost\\ Validation}
            \ar[r] & *+[F.:purple:dashed]\txt{type: Coin\\ to: Transaction\\ Builder\\ asset: BASE\_ASSET\\ amount:\\ $\text{max\_tx\_cost} - \text{network fee}$} \\
        & \phantom{*+[F=]\txt{Placeholder}}
            & \phantom{*+[F=]\txt{Placeholder}}
            & *+[F.:grey:dashed]\txt{type: Change\\ to: master\\ asset: BASE\_ASSET\\ amount: \\ $\text{receiver value} - \text{network fee} - \text{network fee}$}
    }}
    }
    \caption{Compact Module 01 Legacy EVM Transaction Flow and Verification}
    \label{fig:legacy-flow-xy-compact}
\end{figure}


\subsection{Security Properties}

\subsubsection{Asset Conservation}
For module01 asset $M_1$ and predicate address $P$:
\[ \exists \text{ output } o : o.asset = M_1 \land o.amount = 1 \land o.to = P \]

\subsubsection{Signature Replay Prevention}
The nonce mechanism in Module 01 serves primarily as protection against signature replay attacks rather than traditional transaction
sequencing. For a given transaction signature with nonce value $t$, the system maintains a monotonically decreasing counter $n$:
\[ n = NONCE\_MAX - t \]
\[ \exists \text{ output } o : o.asset = N \land o.amount = n - 1 \]

This ensures each signature can only be used once, as the nonce asset's decreasing value prevents reuse of any previously
signed transaction data.

\subsubsection{Value Transfer}
For transaction value $v$ and max cost $c$:
\[ \sum \text{inputs} \geq v + c \]
\[ \exists \text{ output } o : o.asset = BASE\_ASSET \land o.amount = v \land o.to = receiver \]


\subsection{Error Handling}
The module includes error handling that propagates errors to at maximum the \text{main()} scope. However as predicates
must return boolean results, all errors result in a \text{false} returned from the main execution.

