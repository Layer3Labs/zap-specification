\newpage
\section{Module 03 - {\ttfamily ""module03\_erc20"}}
\label{sec:module03_predicate}

\subsection{Overview}
Module 03 enables Fuel native assets to be transferred using the familiar ERC20 token transfer pattern. It processes EIP-1559 EVM transactions
that contain ERC20 transfer function calls by mapping the ERC20 contract addresses in these transactions to their corresponding Fuel native
AssetIds (see section \ref{subsec:asset_maapping:sec:module03_predicate} for detailed mapping). This allows users to interact with Fuel
native assets using standard EVM tooling and interfaces while maintaining the security properties of the ZapWallet architecture. \\

For a transaction $T$ with recipient EVM address $r_{evm}$ and ERC20 contract address $c_{evm}$, the transfer is mapped:
\[ f: (c_{evm}, r_{evm}) \rightarrow (a_{fuel}, w) \]
where:
\begin{itemize}
    \item $a_{fuel}$ is the corresponding Fuel native AssetId
    \item $w$ is the recipient's ZapWallet address derived as described in section \ref{sec:module01_predicate}
\end{itemize}




% % ----------------------------------------------------------------------------------------------
% ERC20 --> Native asset transaction.

\subsubsection{ERC20 Address to Fuel AssetId Mapping}
\label{subsec:asset_maapping:sec:module03_predicate}
The ERC20 contract address from the RLP transaction data must be mapped to its corresponding Fuel native AssetId. This
mapping is achieved through a 20-byte comparison mechanism: \\

\textbf{Address Structure:}
\begin{itemize}
    \item EVM addresses are 20 bytes
    \item Fuel AssetIds are 32 bytes (use of AssetId or b256 types vary)
\end{itemize}

For a given ERC20 contract address $a_{evm}$ and Fuel AssetId $a_{fuel}$, the matching is performed as:

\[ a_{fuel} = \underbrace{0^{12}}_\text{prefix} \parallel \underbrace{a_{evm}}_\text{20 bytes} \]

The comparison function verifies:
\begin{small}
\begin{verbatim}
x: [0x000000000000000000000000][aaaaaaaaaaaaaaaaaaaaaaaaaaaaaaaaaaaaaaaaa]
                12 zeros             20-byte EVM address
y: [aaaaaaaaaaaaaaaaaaaaaaaaaaaaaaaaaaaaaaaaa][bbbbbbbbbbbbbbbbbbbbbbbbbbbbbbbb]
        20-byte EVM address                    12 arbitrary bytes
\end{verbatim}
\end{small}

where:
\begin{itemize}
    \item $x$ is the expected AssetId layout (32 bytes)
    \item $y$ is the full Fuel AssetId being compared
    \item The function compares the last 20 bytes of $x$ with the first 20 bytes of $y$
\end{itemize}

This mapping ensures that ERC20 transfers in EVM transactions can be correctly matched to their corresponding Fuel native
assets while maintaining the full 32-byte precision required by the Fuel network.\\






% % ----------------------------------------------------------------------------------------------
% % ----------------------------------------------------------------------------------------------
% Module 03 functionality

\subsection{Core Functionality}
The module validates transactions through several key steps:
\begin{enumerate}
    \item ERC20 transfer data extraction from EIP-1559 transaction
    \item ERC20 contract address to Fuel AssetId mapping
    \item Signature and chain ID verification
    \item Asset input and output validation
    \item Nonce management
    \item Gas price and limit verification
\end{enumerate}

\subsection{Asset Flow}
The module processes four types of assets:
\begin{itemize}
    \item Module 03 asset (for validation authorization)
    \item Nonce asset (to prevent signature replay)
    \item Fuel Native asset $\neq$ \text{BASE\_ASSET} (mapped from ERC20 contract) for transfer
    \item \text{BASE\_ASSET} (for gas payment and builder tip) from the master.
\end{itemize}

In the same way as Module 01 and 02, the builder\_tip needs to exist as a Coin Output with some value (can be zero). It is not enforced that the transaction builder
is required to "tip" themselves. See section \ref{subsec:gas_costing:sec:module02_predicate} for details.


\subsection{Transaction Structure}
\begin{figure}[H]
    \scalebox{0.85}{
    \centerline{
    \xymatrix@R=2.5pc@C=1.5pc{
        & \txt{Input type\\ Asset type\\ Amount}
        & \txt{Verification\\ Signature \\ type}
        & \txt{Output type\\ Receiver\\ Asset type\\ Amount} \\
        & *+[F]\txt{Coin predicate:\\ Module 03\\ asset: Module 03\\ AssetId\\ amount: 1}
            \ar[r] & *+[H]\txt{EIP-1559\\ ERC20\\ Transfer}
            \ar[r] & *+[F]\txt{type: Coin\\ to: Module 03\\ asset: Module 03\\ AssetId\\ amount: 1} \\
        & *+[F=]\txt{Coin predicate:\\ Master\\ asset: Nonce AssetId\\ amount: $n$,\\ where $n \geq 1$}
            \ar[r] & *+[H]\txt{Nonce value \\ \& ownership\\ validation}
            \ar[r] & *+[F=]\txt{type: Coin\\ to: Master\\ asset: Nonce AssetId\\ amount: $n - 1$} \\
        & *+[F=]\txt{Coin predicate:\\ Master\\ asset: Native Asset\\ amount: token\_value}
            \ar[r] & *+[H]\txt{Token Value \\ Validation}
            \ar[r] & *+[F]\txt{type: Coin\\ to: Receiver\\ asset: Native Asset\\ amount: token\_value} \\
        & *+[F]\txt{Coin predicate:\\ Master\\ asset: BASE\_ASSET\\ amount: max\_tx\_cost}
            \ar[r] & *+[H]\txt{Gas \\ Validation}
            \ar[r] & *+[F.:purple:dashed]\txt{type: Coin\\ to: Transaction\\ Builder\\ asset: BASE\_ASSET\\ amount:\\ $\text{max\_tx\_cost} - \text{network fee}$} \\
        & \phantom{*+[F]\txt{Placeholder}}
            & \phantom{*+[H]\txt{Placeholder}}
            & *+[F=]\txt{type: Change\\ to: Master\\ asset: Native Asset\\ amount: remaining}
    }}
    }
    \caption{Module 03 ERC20 Transaction Flow and Verification}
    \label{fig:erc20-flow-xy}
\end{figure}

\subsection{Security Properties}

\subsubsection{Asset Conservation}
For module03 asset $M_3$ and predicate address $P$:
\[ \exists \text{ output } o : o.asset = M_3 \land o.amount = 1 \land o.to = P \]

\subsubsection{Native Asset Conservation}
For native asset $a$ mapped from ERC20 contract:
\[ \sum \text{inputs}(a) = \sum \text{outputs}(a) \]


\subsubsection{Signature Replay Prevention}
The nonce mechanism in Module 03 serves primarily as protection against signature replay attacks rather than traditional transaction
sequencing. For a given transaction signature with nonce value $t$, the system maintains a monotonically decreasing counter $n$:
\[ n = NONCE\_MAX - t \]
\[ \exists \text{ output } o : o.asset = N \land o.amount = n - 1 \]

This ensures each signature can only be used once, as the nonce asset's decreasing value prevents reuse of any previously
signed transaction data.

\subsection{Error Handling}
The module includes error handling that propagates errors to at maximum the \text{main()} scope. However as predicates
must return boolean results, all errors result in a \text{false} returned from the main execution.



